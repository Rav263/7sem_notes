\documentclass[a4paper, 12pt, titlepage, finall]{extreport}

%различные пакеты

\usepackage[T1, T2A]{fontenc}
\usepackage[russian]{babel}
\usepackage{tikz}
\usepackage{geometry}
\usepackage{indentfirst}
\usepackage{fontspec}
\usepackage{graphicx}
\usepackage{listings}
\usepackage{array}
\graphicspath{{./images/}}

\usetikzlibrary{positioning, arrows}

\geometry{a4paper, left = 15mm, top = 10mm, bottom = 20mm, right = 15mm}

\setmainfont{Spectral Light}%{Times New Roman}
\setmonofont{DejaVu Sans Mono}
%\setcounter{secnumdepth}{3}
%\setcounter{tocdepth}{3}

\lstdefinestyle{customc}{
  belowcaptionskip=1\baselineskip,
  breaklines=true,
  frame=L,
  xleftmargin=\parindent,
  language=C,
  showstringspaces=false,
  basicstyle=\footnotesize\ttfamily,
  keywordstyle=\bfseries\color{green!40!black},
  commentstyle=\itshape\color{purple!40!black},
  identifierstyle=\color{blue},
  stringstyle=\color{orange},
}

\lstdefinestyle{customasm}{
  belowcaptionskip=1\baselineskip,
  frame=L,
  xleftmargin=\parindent,
  language=[x86masm]Assembler,
  basicstyle=\footnotesize\ttfamily,
  commentstyle=\itshape\color{purple!40!black},
}

\lstset{escapechar=@,style=customc}

\begin{document}

    \begin{titlepage}
        \begin{center}
            {\small \sc Московский государственный университет имени М.~В.~Ломоносова\\
            Факультет вычислительной математики и кибернетики\\
            Кафедра автоматизации систем вычислительных комплексов\\}
            \vfill
            {\large \sc Конспекты лекций.}\\~\\

            {\large \bf Распределённые системы.}\\~\\

        \end{center}
        
        \begin{flushright}
            \vfill
            \vfill
            {Никифоров Никита Игоревич, 421 группа}\\
            {Лектор:}\\
            {профессор Крюков Виктор Алексеевич, доцент Бахтин Владимир Александрович}\\
        \end{flushright}

        \begin{center}
            \vfill
            {\small Москва\\2020}
        \end{center}
    \end{titlepage}

    \chapter*{Аннотация}
    \newpage
    \tableofcontents
    \newpage
    \chapter*{Введение}
        \addcontentsline{toc}{chapter}{\protect\numberline{}Введение}%
        
    \chapter{Лекция 1}
        \section{Основные понятия}
            \textbf{Опр 1.} Расспределённая компьютерная система $-$ совокупность связанных сетью независимых компьютеров, 
            которая предоставляется пользователю единым компьютером.\\
            \textbf{Опр 2.} Расспределённая программная система $-$ совокупность компонентов, взаимодействующих посредством обмена сообщениями\\
            Поговорим об особенностях распределённых систем (PC):
            \begin{enumerate}
                \item Конкуретность
                \item Отсутствие глобальных часов $-$ невозможно полагаться на то, что все будут знать точное время.
                \item Независимые отказы $-$ некоторые машины могут выходить из строя, при этом выход какой-либо машины не влияет на другие.
            \end{enumerate}
            Поговорим о тенденциях определяющих развитие РС сегодня:
            \begin{enumerate}
                \item Широкое распространение сетевых технологий
                \item Повсеместное использование компьютинга в сочетании с желание поддержать мобильность пользователей в РС.
                \item Растущий спрос на мультимедийные услуги.
                \item Предоставление РС как утилиты (арендовать а не покупать ресурсы)
            \end{enumerate}
            Какие есть проблемы в расспределённых системах?
            \begin{enumerate}
                \item гетерогенность $-$ в большой РС ресурсы могут быть различны (различные ядра процессоров).
                \item Открытость $-$ использование одних стандартов на различных машинах.
                \item Секретность $-$ вещь понятная, но сложная.
                \item Матабируемость (ресурсная, географическая, организационная).
                \item Надёжность $-$ Должно быть надёжно.
                \item Конкурентность $-$ Ыыыыы.
                \item Прозрачноть $-$ Пользователь не должен знать ничего.
                \item Качество обслуживания
            \end{enumerate}
        \section{История ОС}
            \textbf{Опр 3.} Персональные ЭВМ $-$ пультовой режим\\
            В использовании были огромные ЭВМ, в которых были библиотеки программ ввода$-$вывода, которые являлись служебными программами.
            Изначально программистами были учёные, которые были специалиставми в различных областях, но потом они делигировали
            взаимодествие с ЭВМ специально обученным людям $-$ операторам.
            В ОС должны быть реализованы следующие вещи, для реализации мультипрограммного режима:
            \begin{itemize}
                \item Механизм прерываний
                \item Таймер
                \item Привелигерованный режим
                \item Защита памяти
            \end{itemize}
            А есть ли возможность обеспечить мультипрограммный режим без использования данных механизмов?
            Как пример операционная система \textbf{Pick}. ПРОСТО РЕАЛИЗУЙ ЭТО ПРОГРАММНО.
\end{document}
