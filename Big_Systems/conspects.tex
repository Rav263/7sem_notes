\documentclass[a4paper, 12pt, titlepage, finall]{extreport}

%различные пакеты

\usepackage[T1, T2A]{fontenc}
\usepackage[russian]{babel}
\usepackage{tikz}
\usepackage{geometry}
\usepackage{indentfirst}
\usepackage{fontspec}
\usepackage{graphicx}
\usepackage{listings}
\usepackage{array}
\graphicspath{{./images/}}

\usetikzlibrary{positioning, arrows}

\geometry{a4paper, left = 15mm, top = 10mm, bottom = 20mm, right = 15mm}

\setmainfont{Spectral Light}%{Times New Roman}
\setmonofont{DejaVu Sans Mono}
%\setcounter{secnumdepth}{3}
%\setcounter{tocdepth}{3}

\lstdefinestyle{customc}{
  belowcaptionskip=1\baselineskip,
  breaklines=true,
  frame=L,
  xleftmargin=\parindent,
  language=C,
  showstringspaces=false,
  basicstyle=\footnotesize\ttfamily,
  keywordstyle=\bfseries\color{green!40!black},
  commentstyle=\itshape\color{purple!40!black},
  identifierstyle=\color{blue},
  stringstyle=\color{orange},
}

\lstdefinestyle{customasm}{
  belowcaptionskip=1\baselineskip,
  frame=L,
  xleftmargin=\parindent,
  language=[x86masm]Assembler,
  basicstyle=\footnotesize\ttfamily,
  commentstyle=\itshape\color{purple!40!black},
}

\lstset{escapechar=@,style=customc}

\begin{document}

    \begin{titlepage}
        \begin{center}
            {\small \sc Московский государственный университет имени М.~В.~Ломоносова\\
            Факультет вычислительной математики и кибернетики\\
            Кафедра автоматизации систем вычислительных комплексов\\}
            \vfill
            {\large \sc Конспекты лекций.}\\~\\

            {\large \bf Проектирование и программирование большиех систем на C++.}\\~\\

        \end{center}
        
        \begin{flushright}
            \vfill
            \vfill
            {Никифоров Никита Игоревич, 421 группа}\\
            {Лектор:}\\
            {Коноводов Виталий Александрович}\\
        \end{flushright}

        \begin{center}
            \vfill
            {\small Москва\\2020}
        \end{center}
    \end{titlepage}

    \chapter*{Аннотация}
        Форма контроля:
        \begin{itemize}
            \item Нужно набирать баллы
            \item Домашние задания
            \item Тесты (Дополнительные баллы)
            \item Экзамен
            \item + Дополнительные домашки (засчитываются только первым 2-3 людям)
        \end{itemize}
        E-mail лектора: vkonovodov@gmail.com
    \newpage
    \tableofcontents
    \newpage
    \chapter*{Введение}
        \addcontentsline{toc}{chapter}{\protect\numberline{}Введение}%
    \chapter{Лекция 1}
        \section{Шаблоны в C++}
            Шаблоны $-$ замечательная вещь. Пример:
\begin{lstlisting}
template <typename T>
void f(T& param);

int x = 1;
const int cx = x;
const int& rx = x;

f(x);  // T - int
f(cx); // T - const int
f(rx); // T - const int&
\end{lstlisting}
            Существует два правила:
            \begin{itemize}
                \item Случай 1 $-$ тип параметра указатель или ссылка. Ссылочная часть игнорируется.
                \item Случай 2 $-$ передача по значению. Ссылочная часть игнорируется. Если после этого остаётся const, то он тоже игнорируется.
            \end{itemize}
\begin{lstlisting}
template <typename T>
void f(T param);

const char* const ptr = "abcd"; // Костантный указатель на константу.
f(ptr); // param - const char*
\end{lstlisting}

            Указатели на массивы. Типы s и p различные.

\begin{lstlisting}
const char s[] = "ab"; // const char[3]
const char* p = s; // const char*

template <typename T>
void f(T param);
void g(T& param);

f(s); // const char *
g(s); // const char (&) [3]
f(p); // const char*&
\end{lstlisting}

            Ещё немного примеров и лайфхаков. Как узнать тип функции? Просто не опиши тело функции и вызови её.
            Компилятор сам тебе скажет, что там за тип.
        \section{Автоматические вывод типа переменной}
\begin{lstlisting}
auto x = 0; // int
auto y = x, *z = &x; // int, int*
auto& ref = x; // int&
const auto &xcref = x; // const int&
auto *p = &x; // int *p
const auto* cp = p; // const int *p
const auto pc = &x; // int * const
\end{lstlisting}
            Всегда пишите * после {\bf auto} если нужен указатель! Это не сложно, но добавит понимания.
            Поговорим про {\bf initializer\_list}
\begin{lstlisting}
#include <initializer_list>

int main() {
    int x = 2;
    int y(2);

    // 4 способа чяерез auto
    auto x1 = 2; // точно int
    auto x2(2); // точно int
    auto x3 = {2}; // Что-то что не int (std::initializer_list<int>)
    auto x4{2}; // int
}
\end{lstlisting}
            Если для {\bf auto} аргумент указан в фигурных скобках, то для auto это будет {\bf initializer\_list}. 
            И это единственное отличие автоматического выбора типов для шаблонов и auto.
            Для {\bf template} это вообще не будет работать.
        \section{Ключевое слово decltype}
\begin{lstlisting}
decltype(1 + 2) x = 1; // int
decltype(x) y = x; // int
decltype(1, x) z = x; // int & (Потому что , тоже оператор, который возвращает ссылку)


auto cmp = [](char const* p1, char const* p2) {
    return strcmp(p1, p2) < 0;
}

std::map<char const*, unsigned, decltype(cmp)> MyMap(cmp);
\end{lstlisting}
        \section{Круглые и фигурные скобки}
        До С++11 невозможно было создать контейнер содержащим определённый набор значений.
\begin{lstlisting}
std::vector<int> v{1, 3, 5};

double a, b;
int c{a + b}; // error!
int d = a + b; // ok!
int e(a + b); // ok!

\end{lstlisting}

\begin{lstlisting}
class T {
    public:
    T(int a, bool b);
    T(int a, double b);
    T(std::initializer_list<long double> l);
};

int main() {
    T t1(10, true)  // первый int, bool
    T t2{10, true}; // третий init list
    T t3(10, 0.2);  // второй int, double
    T t4{10, 0.2};  // третий intl list
    T t6({}); // Не будет работать!!
}
\end{lstlisting}
        Подумать, как передавать пустой {\bf initializer\_list} конструктору.
\begin{lstlisting}
for (T var : container) {
    // ....
}

// begin и end зопоминаются перед циклом.
using TElem = std::pair<int, int>
using TCont = std::vector<TElem>;
Tcont container;

for (auto x : container) {
    // ...
}
\end{lstlisting}
        \section{Ключевое слово override}
\begin{lstlisting}
struct A{
    virtual void foo();
    void bar();
};

struct B :A{
    void foo() const override; // ошибка
    void foo() override;
    void bar() override; // ошибка
}

\end{lstlisting}
        \section{Ключевое слово final}
            Если указано {\bf final} то нельзя наследоваться дальше.
\begin{lstlisting}
struct A {
    virtual void foo() final; // Запрет переопределения функции
    void bar() final; // ошибка, так как функция не виртуальная
};

struct B final : A { // От структуры B нельзя отнаследоваться
    void foo(); //ошибка т.к. foo final
};

struct C : B { // Ошибка т.к. B - final 
};
\end{lstlisting}
\end{document}
