\documentclass[a4paper, 12pt, titlepage, finall]{extreport}

%различные пакеты

\usepackage[T1, T2A]{fontenc}
\usepackage[russian]{babel}
\usepackage{tikz}
\usepackage{geometry}
\usepackage{indentfirst}
\usepackage{fontspec}
\usepackage{graphicx}
\usepackage{listings}
\usepackage{array}
\graphicspath{{./images/}}

\usetikzlibrary{positioning, arrows}

\geometry{a4paper, left = 15mm, top = 10mm, bottom = 20mm, right = 15mm}

\setmainfont{Spectral Light}%{Times New Roman}
\setmonofont{DejaVu Sans Mono}
%\setcounter{secnumdepth}{3}
%\setcounter{tocdepth}{3}

\lstdefinestyle{customc}{
  belowcaptionskip=1\baselineskip,
  breaklines=true,
  frame=L,
  xleftmargin=\parindent,
  language=C,
  showstringspaces=false,
  basicstyle=\footnotesize\ttfamily,
  keywordstyle=\bfseries\color{green!40!black},
  commentstyle=\itshape\color{purple!40!black},
  identifierstyle=\color{blue},
  stringstyle=\color{orange},
}

\lstdefinestyle{customasm}{
  belowcaptionskip=1\baselineskip,
  frame=L,
  xleftmargin=\parindent,
  language=[x86masm]Assembler,
  basicstyle=\footnotesize\ttfamily,
  commentstyle=\itshape\color{purple!40!black},
}

\lstset{escapechar=@,style=customc}

\begin{document}

    \begin{titlepage}
        \begin{center}
            {\small \sc Московский государственный университет имени М.~В.~Ломоносова\\
            Факультет вычислительной математики и кибернетики\\
            Кафедра автоматизации систем вычислительных комплексов\\}
            \vfill
            {\large \sc Конспекты лекций.}\\~\\

            {\large \bf Конструирование компиляторов.}\\~\\

        \end{center}
        
        \begin{flushright}
            \vfill
            \vfill
            {Никифоров Никита Игоревич, 421 группа}\\
            {Лектор:}\\
            {профессор }\\
        \end{flushright}

        \begin{center}
            \vfill
            {\small Москва\\2020}
        \end{center}
    \end{titlepage}

    \chapter*{Аннотация}
    \newpage
    \tableofcontents
    \newpage
    \chapter*{Введение}
        \addcontentsline{toc}{chapter}{\protect\numberline{}Введение}%
        
    \chapter{Неоптимизирующий компилятор}
        Структура неоптимизируещего компилятора:
        \begin{itemize}
            \item Передний план (frontend) $-$ получает исходный код и переводит его в специальных вид \textbf{HIR}.
            \item Задний план (backend) $-$
        \end{itemize}
        \textbf{Опр 1.} HIR $-$ атрибутированное абстрактное синтаксическое дерево (дерево программы).\\
        \textbf{Опр 2.} Исходный код $-$ текст моуля компиляции в виде последовательности символов.\\
        \textbf{Опр 3.}
        Анализ исходного кода:
        \begin{itemize}
            \item Лексический анализ
            \item Синтаксический анализ
            \item Семантический анализ
        \end{itemize}
        Синтаксический анализ переводит исходный код программы в набор токенов, его цель сократить программу, путём удаления ненужных символов.
        Лексический анализатор $-$ представляет из себя конечный автомат.
\end{document}
